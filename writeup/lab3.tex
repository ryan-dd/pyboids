\documentclass[12pt]{article}

\begin{document}

\section*{Nearest Neighbors}

\subsection*{Swarm}

For the model in the paper, the following parameters produced a swarm.

\begin{itemize}
  \item{Zone of Repulsion: 1}
  \item{Zone of Orientation: .1}
  \item{Zone of Attraction: 10}
  \item{Tau: 4}
  \item{Limit Angle: $\pi$}
\end{itemize}

Directly applying these results with the 5 nearest neighbors produced an interesting dynamic. Very quickly 
all the points except one or two would colapse very close to each other. Then this cluster and the small point would 
move around keeping a fixed distance and oscilating back and forth.

By reducing tau to 1 I was able to recreat a swarm. The difference is that most the area of the swarm was smaller
and there were also 4 or 5 points that were farther away interacting.

\subsection*{Torus}

For the model in the paper, the following parameters produced a torus.

\begin{itemize}
  \item{Zone of Repulsion: 1}
  \item{Zone of Orientation: .1}
  \item{Zone of Attraction: 5}
  \item{Tau: 1}
  \item{Limit Angle: $\pi / 4$}
\end{itemize}

With these settings the NN approach is interesting. There forms one main group that moves around like a swarm. 
The rest are swarming fairly uniformally over a loarge area. The small number of agents they exchange information with 
has swarm techniques at a much broader scale. When the big cluster approaches these some agents join and some leave. Overall
the same effect is maintained.

This was the hardest to get to match the results. In the end I was able to get a noisy torus. There would be a clear torus 
and then it would colapse and reform. To do this I removed the limit angle.

 
\subsection*{Highly Parallel Group}

For the model in the paper, the following parameters produced a highly parallel group.

\begin{itemize}
  \item{Zone of Repulsion: 1}
  \item{Zone of Orientation: 15}
  \item{Zone of Attraction: 15}
  \item{Tau: 1}
  \item{Limit Angle: $\pi / 4$}
\end{itemize}

These parameters also produce the same general effect with nearest neighbors. The only difference is sometimes
there is a little more variance and sensitivity to noise.

\subsection*{Dynamic Parallel Group}

For the model in the paper, the following parameters produced a dynamic parallel group.

\begin{itemize}
  \item{Zone of Repulsion: 1}
  \item{Zone of Orientation: 5}
  \item{Zone of Attraction: 15}
  \item{Tau: 1}
  \item{Limit Angle: $\pi / 4$}
\end{itemize}

When using these parameters it started off very similar to the highly parallel group. The points moved very quickly as a flock
together maintaining realativly the same formation. What was interesting here was that as time went on the stability of the
group began to falter. Some of the points slowly got further away unitl were behaving differntly from the group. As this happend 
the circle that was typical began to deform into a streched elipse.
 
\end{document} 
  
